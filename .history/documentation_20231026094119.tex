\documentclass{article}

\usepackage{amsmath}
\usepackage{tikz}

\title{Turing Machine Documentation}
\author{Your Name}
\date{\today}

\begin{document}

\maketitle

\section{Introduction}
This document provides documentation for a Turing machine, which is a mathematical model of computation. The purpose of this machine is to perform computations on an input tape, which is a sequence of symbols.

\section{Definition}
A Turing machine is defined by a 7-tuple $(Q, \Sigma, \Gamma, \delta, q_0, q_{accept}, q_{reject})$, where:
\begin{itemize}
    \item $Q$ is a finite set of states.
    \item $\Sigma$ is a finite set of input symbols.
    \item $\Gamma$ is a finite set of tape symbols, where $\Sigma \subseteq \Gamma$.
    \item $\delta$ is the transition function, which maps $Q \times \Gamma$ to $Q \times \Gamma \times \{L, R\}$, where $L$ and $R$ represent left and right movement on the tape.
    \item $q_0$ is the initial state.
    \item $q_{accept}$ is the accepting state.
    \item $q_{reject}$ is the rejecting state.
\end{itemize}

\section{States and Transitions}
The Turing machine operates by moving between states and performing transitions on the tape. The states and transitions are defined by the 7-tuple above.

\section{Examples}
Here are some examples of input and output for the Turing machine:
\begin{itemize}
    \item Input: 0011, Output: 1100
    \item Input: 101, Output: 010
\end{itemize}

\section{Conclusion}
The Turing machine is a powerful tool for performing computations on input tapes. It has applications in computer science, mathematics, and other fields.

\end{document}