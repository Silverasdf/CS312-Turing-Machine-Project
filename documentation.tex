\documentclass{article}

\usepackage{amsmath}
\usepackage{tikz}

\title{Turing Machine Documentation}
\author{Ryan Peruski, Maria Hernandez}
\date{\today}

\begin{document}

\maketitle

\section{Introduction}
This document provides documentation for a Turing machine, which is a mathematical model of computation. The purpose of this machine is to perform computations on an input tape, which is a sequence of symbols.

\section{Definition}
A Turing machine is meant to make a stack and queue depending on the input

\section{States and Transitions}
The Turing machine operates by moving between states and performing transitions on the tape. The states and transitions are defined by the 7-tuple above.

\begin{itemize}
    \item $Red$ are all the reject states ($q_{reject}$ )
    \item $Green$ are all the accept states ($q_{accept}$ )
    \item $Blue$ is the Queue
    \item $Magenta$ is the Stack
    \item $Black$ are the initial states to set up the $\#$
\end{itemize}

\section{Examples}
Here are some examples of input and output for the Turing machine:
\begin{itemize}
    \item \begin{verbatim} Input: SA1A2D, Output: xxxxxx#1 \end{verbatim}
    \item \begin{verbatim} Input: QA1A2D, Output: xxxxxx#2 \end{verbatim}
    \item \begin{verbatim} Input: QA#A2D, Output: INVALID \end{verbatim}
\end{itemize}

\section{Conclusion}
The Turing machine is a powerful tool for performing computations on input tapes. It has applications in computer science, mathematics, and other fields.

\end{document}